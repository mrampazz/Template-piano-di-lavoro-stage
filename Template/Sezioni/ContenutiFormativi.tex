\section*{Contenuti formativi previsti}
% Personalizzare indicando le tecnologie e gli ambiti di interesse dello stage
Durante il progetto di stage lo studente avrà occasione di approfondire le sue conoscenze nell’ambito della realizzazione di interfacce grafiche web tramite tecniche di programmazione funzionale.\\
Il progetto di stage include lo studio e l'applicazione delle seguenti tecnologie:
\begin{enumerate}
	\item \textbf{Kotlin}: linguaggio di programmazione fortemente tipizzato, multi-paradigma e con multipli backend, supportato da un ottimo ambiente di sviluppo open source (Intellij IDEA Community Edition.);
	\item \textbf{ReactJS e Redux}: framework complementari per la realizzazione di interfacce web interattive, che favoriscono l'utilizzo di tecniche di programmazione funzionale, strutture dati immutabili e la suddivisione del lavoro in moduli coesi e disaccoppiati;
	\item \textbf{SCSS}: linguaggio che permette di definire gli stili grafici di interfacce web complesse tramite l'applicazione delle tecniche di progettazione ad oggetti (ereditarietà, polimorfismo.).
\end{enumerate}
%\newpage