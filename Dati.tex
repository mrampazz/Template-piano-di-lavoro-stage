%----------------------------------------------------------------------------------------
%   USEFUL COMMANDS
%----------------------------------------------------------------------------------------

\newcommand{\dipartimento}{Dipartimento di Matematica ``Tullio Levi-Civita''}

%----------------------------------------------------------------------------------------
% 	USER DATA
%----------------------------------------------------------------------------------------

% Data di approvazione del piano da parte del tutor interno; nel formato GG Mese AAAA
% compilare inserendo al posto di GG 2 cifre per il giorno, e al posto di 
% AAAA 4 cifre per l'anno
\newcommand{\dataApprovazione}{Data}

% Dati dello Studente
\newcommand{\nomeStudente}{Marco}
\newcommand{\cognomeStudente}{Rampazzo}
\newcommand{\matricolaStudente}{1170754}
\newcommand{\emailStudente}{marco.rampazzo.10@studenti.unipd.it}
\newcommand{\telStudente}{+ 39 347 99 27 885}

% Dati del Tutor Aziendale
\newcommand{\nomeTutorAziendale}{Tobia}
\newcommand{\cognomeTutorAziendale}{Conforto}
\newcommand{\emailTutorAziendale}{tobia.conforto@gruppo4.it}
\newcommand{\telTutorAziendale}{ + 39 347 39 07 311}
\newcommand{\ruoloTutorAziendale}{}

% Dati dell'Azienda
\newcommand{\ragioneSocAzienda}{GRUPPO 4 S.r.l.}
\newcommand{\indirizzoAzienda}{Via Carducci, 24 Padova}
\newcommand{\sitoAzienda}{www.gruppo4.it}
\newcommand{\emailAzienda}{mail@mail.it}
\newcommand{\partitaIVAAzienda}{P.IVA 12345678999}

% Dati del Tutor Interno (Docente)
\newcommand{\titoloTutorInterno}{Prof.}
\newcommand{\nomeTutorInterno}{Claudio Enrico}
\newcommand{\cognomeTutorInterno}{Palazzi}

\newcommand{\prospettoSettimanale}{
     % Personalizzare indicando in lista, i vari task settimana per settimana
     % sostituire a XX il totale ore della settimana
    \begin{itemize}
        \item \textbf{Prima Settimana}
        \begin{itemize}
            \item presentazione del progetto e dei contenuti formativi previsti;
            \item studio iniziale autonomo delle tecnologie, tramite tutorial e manuali online;
            \item impostazione di un ambiente di sviluppo.
        \end{itemize}
        \item \textbf{Seconda Settimana} 
        \begin{itemize}
            \item definizione dell'architettura della soluzione, ovvero della suddivisione del progetto in moduli software disaccoppiati;
            \item definizione degli obiettivi per gli sprint di sviluppo;
            \item impostazione iniziale del codice sorgente del progetto;
            \item ulteriore studio delle tecnologie.
        \end{itemize}
        \item \textbf{Terza Settimana} 
        \begin{itemize}
            \item primo sprint di sviluppo;
            \item eventuale revisione dell'architettura.
        \end{itemize}
        \item \textbf{Quarta Settimana} 
        \begin{itemize}
            \item secondo sprint di sviluppo, inclusi unit test.
        \end{itemize}
        \item \textbf{Quinta Settimana} 
        \begin{itemize}
            \item terzo sprint di sviluppo, inclusi unit test.
        \end{itemize}
        \item \textbf{Sesta Settimana} 
        \begin{itemize}
            \item sprint finale di sviluppo, inclusi test di integrazione.
        \end{itemize}
        \item \textbf{Settima Settimana} 
        \begin{itemize}
            \item finalizzazione degli unit test;
            \item finalizzazione degli stili grafici;
        \end{itemize}
        \item \textbf{Ottava Settimana} 
        \begin{itemize}
            \item finalizzazione dei test di integrazione;
        \end{itemize}
    \end{itemize}
}

% Indicare il totale complessivo (deve essere compreso tra le 300 e le 320 ore)
\newcommand{\totaleOre}{}

\newcommand{\obiettiviObbligatori}{
	 \item \underline{\textit{O01}}: progettazione del componente di grafica avanzata per realizzare tabelle Pivot interattive;
	 \item \underline{\textit{O02}}: realizzazione del componente mediante ReactJS (Redux) e SCSS;
	 \item \underline{\textit{O03}}: realizzazione del collegamento tra il backend AJAX e la soluzione con Kotlin;
	 \item \underline{\textit{O04}}: realizzazione di una suite di test d'unità e d'integrazione;
	 
}

\newcommand{\obiettiviDesiderabili}{
	 \item \underline{\textit{D01}}: implementazione di un workflow di Continous Integration (github actions / jenkins);
	 \item \underline{\textit{D02}}: realizzazione dei componenti grafici mediante tecniche di programmazione funzionale;
}